% !TEX encoding = UTF-8 Unicode
\documentclass{../headers/td_upc}
\makeatletter
%--------------------------------------------------------------------------------

%\usepackage[french]{babel}
\usepackage[a4paper,hmargin=20mm,vmargin=25mm]{geometry}
\usepackage{dsfont}
\usepackage[utf8]{inputenc}
\usepackage[T1]{fontenc}
\usepackage{lmodern}

%-------------------------------- AMS --------------------------------------%
\usepackage{amsmath}
\usepackage{amsbsy}
\usepackage{amsfonts}
\usepackage{amssymb}
\usepackage{amscd}
\usepackage{amsthm}


\theoremstyle{definition}
%\newtheorem{defi}{Definition}[section]
\newtheorem{defi}{Definition}[]
\newtheorem{remark}[defi]{Remarque}
\newtheorem{prop}[]{Proposition}
\newtheorem{theo}[]{Th\'eor\`eme}



\usepackage{xcolor}
\usepackage{mathtools}
\usepackage{eurosym}
\usepackage{nicefrac}
\usepackage{latexsym}

\usepackage{multicol}
\usepackage[inline]{enumitem}
\setlist{nosep}
\setlist[itemize,1]{,label=$-$}

\usepackage{sectsty}
%\sectionfont{}
\allsectionsfont{\normalfont\sffamily\bfseries\normalsize}

\usepackage{graphicx}
\usepackage{tikz}
\usetikzlibrary{calc,fadings,decorations.pathreplacing,matrix,arrows,decorations.text}
 \usetikzlibrary{patterns}

\newcommand\chideux[1]{#1<=0||(#1!=int(#1))?1/0:x<=0?0.0:exp((0.5*#1-1.0)*log(x)-0.5*x-lgamma(0.5*#1)-#1*0.5*log(2))}
\newcommand\gauss[2]{1/(#2*sqrt(2*pi))*exp(-((x-#1)^2)/(2*#2^2))} 
\newcommand\student[1]{gamma(.5*(#1+1))/(sqrt(#1*pi)*gamma(.5*#1))*((1+x^2/#1)^(-.5*(#1+1)))}

\usepackage[framemethod=tikz]{mdframed}
\mdfdefinestyle{commandline}{leftmargin=.07\linewidth, rightmargin=.07\linewidth,backgroundcolor=gray!20,linewidth=0pt}
\mdfdefinestyle{codeline}{leftmargin=.01\linewidth, rightmargin=.01\linewidth,linewidth=1pt,backgroundcolor=gray!20,linewidth=0pt}


\usepackage{textcomp}
\usepackage{listings}
\lstloadlanguages{R}
\lstset{upquote=true,
	%columns=flexible,
	keepspaces=true,
	breaklines=true,
	breakindent=0pt,
	basicstyle={\small\ttfamily}
	}
\lstset{extendedchars=true,
	literate={é}{{\'e}}1
		 {è}{{\`e}}1
		 {à}{{\`a}}1
		 {ç}{{\c{c}}}1
		 {œ}{{\oe}}1
		 {ù}{{\`u}}1
		 {É}{{\'E}}1
		 {È}{{\`E}}1
		 {À}{{\`A}}1
		 {Ç}{{\c{C}}}1
		 {Œ}{{\OE}}1
		 {Ê}{{\^E}}1
		 {ê}{{\^e}}1
		 {î}{{\^i}}1
		 {ô}{{\^o}}1
		 {û}{{\^u}}1
		 {°}{{\degre}}1
	}

\lstnewenvironment{script}{%
\lstset{language=,
	    aboveskip=0pt,
	belowskip=0pt
	}
}{}
\surroundwithmdframed[style=commandline]{script}

\lstdefinestyle{codeR}{%
	aboveskip=0pt,
	belowskip=0pt,
	%frame=single,
	commentstyle=\color{green!50!black}\itshape,
	language=R,
	keywordstyle=\color{violet!75},
	stringstyle=\color{red},
	showstringspaces=false,
      %  emph={output,input}, emphstyle=\color{brown!75},
      %  emph={maFonction}, emphstyle=\underline,
	}

\lstnewenvironment{codeR}[1][]{\lstset{style=codeR,#1}}{}
\surroundwithmdframed[style=codeline]{codeR}


\usepackage{algorithm}
\usepackage{algpseudocode}


\usepackage{pgfplots}
\usepgfplotslibrary{fillbetween}
\pgfplotsset{compat=newest}
%\usepgfplotslibrary{external} 
%\tikzexternalize[prefix=./output_latex/]
%\DeclareSymbolFont{RalphSmithFonts}{U}{rsfs}{m}{n}
%\DeclareSymbolFontAlphabet{\mathscr}{RalphSmithFonts}
%\def\mathcal#1{{\mathscr #1}}



% ------------------------------------------- Command -------------------------------%
\providecommand{\abs}[1]{\left|#1\right|}
\providecommand{\norm}[1]{\left\Vert#1\right\Vert}
\providecommand{\U}{\mathcal{U}}
\providecommand{\R}{\mathbb{R}}
\providecommand{\Cc}{\mathcal{C}}
\providecommand{\reg}[1]{\mathcal{C}^{#1}}
\providecommand{\1}{\mathds{1}}
\providecommand{\N}{\mathbb{N}}
\providecommand{\Z}{\mathbb{Z}}
\providecommand{\C}{\mathbb{C}}
\providecommand{\F}{\mathbb{F}}
\providecommand{\K}{\mathbb{K}}
\providecommand{\p}{\partial}
\providecommand{\gR}{\textsc{R}}
\providecommand{\one}{\mathds{1}}
%\renewcommand{\E}{\mathbb{E}}
\renewcommand{\P}{\mathbb{P}}
\providecommand{\ie}{\textit{i.e. }}

\newcommand\rst[2]{{#1}_{\restriction_{#2}}}
%Operateur
\providecommand{\abs}[1]{\left\lvert#1\right\rvert}
\providecommand{\sabs}[1]{\lvert#1\rvert}
\providecommand{\babs}[1]{\bigg\lvert#1\bigg\rvert}
\providecommand{\norm}[1]{\left\lVert#1\right\rVert}
\providecommand{\bnorm}[1]{\bigg\lVert#1\bigg\rVert}
\providecommand{\snorm}[1]{\lVert#1\rVert}
\providecommand{\prs}[1]{\left\langle #1\right\rangle}
\providecommand{\sprs}[1]{\langle #1\rangle}
\providecommand{\bprs}[1]{\bigg\langle #1\bigg\rangle}

\DeclareMathOperator{\deet}{Det}
\DeclareMathOperator{\supp}{Supp}
\DeclareMathOperator{\sinc}{sinc}


\usepackage{ifthen}

\newcommand{\eno}[1]{%
	\ifthenelse{\equal{\version}{eno}}{#1}{}%
}
\newcommand{\cor}[1]{%
        \ifthenelse{\equal{\version}{cor}}{
\medskip 

{\small \color{gray} #1}

\medskip 
}{}
}

%------------------------------------------------------------------------------
\DeclareUnicodeCharacter{00A0}{~}
\makeatother



\def\version{eno}
%\def\version{cor}

\usepackage{hyperref}
\ue{MV4AE035}

\providecommand{\T}{\mathbb{T}}
\providecommand{\1}{\mathds{1}}
\title{TD 3 : Sélection de modèle}


\newcommand{\miniscule}{\@setfontsize\miniscule{5}{6}}
%-----------------------------------------------------------------------------
\begin{document}
	\maketitle
	
	\exo{Fisher global}
	\begin{enumerate}
		\item Rappeler la définition des lois du chi-deux, de Student, et de Fisher.
		\item On se place dans le cadre d'une régression linéaire gaussienne multiple
		$Y = X \beta + \varepsilon$.
		Rappeler les expressions et les lois de $\hat \beta$ et $\hat{\sigma}^2$.
		\item En déduire que la statistique de test de Fisher global
		\[
		F = \dfrac{1}{p \hat \sigma^2} \hat \beta^T (X^TX) \hat \beta
		\]
		suit sous $H_0$ une loi de Fisher de paramètre $p$, $n-p$.
		\item Réécrire $F$ en fonction de $Y$ et $\hat Y$.
		\item Que teste cette statistique ? Que peut-on en dire en pratique ?
	\end{enumerate}


	\exo{Fisher emboîté}
	
	\begin{enumerate}
		\item Rappeler le test entre modèles emboîtés et donner la statistique de test $F$
		en fonction de $Y$, $\hat Y$, et $\hat Y_0$.
		Dans quel contexte retrouve-t-on l'expression de l'exercice 1 ?
		\item Réécrire cette quantité en fonction de $SCR$ et $SCR_0$.
		\item Montrer que
		\[
		F = \dfrac{n-p}{q} \dfrac{R^2 - R_0^2}{1-R^2}\,,
		\]
		où $R^2$ et $R_0^2$ sont les coefficients de détermination associés
		respectivement au modèle complet et au modèle emboîté.
		\cor{
		On a :
		\[
		F = \dfrac{n-p}{q} \dfrac{SCR_0 - SCR}{SCR}
		\]
		et
		\[
		R^2 = 1 - \frac{SCR}{SCT} \quad \text{et} \quad R^2_0 = 1 - \frac{SCR_0}{SCT}
		\]
		i.e.
		\[
		SCR = SCT(1 - R^2) \quad \text{et} \quad SCR_0 = SCT(1 - R^2_0)
		\]
		donc:
		\[
		F = \dfrac{n-p}{q} \dfrac{SCT(1 - R^2_0) - SCT(1 - R^2)}{SCT(1 - R^2)}
		= \dfrac{n-p}{q} \dfrac{R^2 - R^2_0}{1 - R^2}
		\]
		}
	\end{enumerate}

	%%%%%%%%%%%%%%%%%%%%%%%%%%%%%%%%%%%%%%%%%%%%%%%%%%%%%%%%%%%%%%%%%%%%%%%%%%%%%%%%
\exo{Tests de Student et de Fisher}
%%%%%%%%%%%%%%%%%%%%%%%%%%%%%%%%%%%%%%%%%%%%%%%%%%%%%%%%%%%%%%%%%%%%%%%%%%%%%%%%

On considère le modèle de régression linéaire classique à $n$ variables et $p$ prédicteurs:
$$
\mathbf{y} = \mathbf{X} \boldsymbol{\beta} + \boldsymbol{\epsilon}.
$$

On souhaite montrer l'équivalence entre les tests de Student et de Fisher pour
tester la nullité du dernier coefficient:
$$
\mathcal{H}_{0}: \beta_{p}=0 \qquad \text{contre} \qquad \mathcal{H}_{1}: \beta_{p} \neq 0.
$$

\begin{enumerate}
%%%%%%%%%%%%%%%%%%%%%%%%%%%%%%%%%%%%%%%%%%%%%%%%%%%%%%%%%%%%%%%%%%%%%%%%%%%%%%%%
  \item Soient $U \sim \mathcal{N}(0, 1)$ et $V \sim \chi^2_k$
  deux variables aléatoires indépendantes (avec $k$ un entier strictement positif).
  Quelle est la loi de $T = \frac{U}{\sqrt{V/k}}$ ?
  Quelle est la loi de $F = T^{2}$ ?
  
  \cor{
  Par définition, $T$ suit une loi de Student $\mathcal{T}_{k}$.
  
  De plus, $W = U^2 \sim \chi^2_1$ par définition.
  Donc $F = T^2 = \frac{W/1}{V/k}$ suit une loi de Fisher $\mathcal{F}^{1}_{k}$.
  }
  
%%%%%%%%%%%%%%%%%%%%%%%%%%%%%%%%%%%%%%%%%%%%%%%%%%%%%%%%%%%%%%%%%%%%%%%%%%%%%%%%
  \item Rappelez les hypothèses classiques du modèle linéaire gaussien.
  Quelles sont les dimensions de $\mathbf{y}$, $\mathbf{X}$, $\boldsymbol{\beta}$ et $\boldsymbol{\epsilon}$ ?
  On se place sous ces hypothèses dans toute la suite.
  
  \cor{
  Voir le cours.
  }
  
%%%%%%%%%%%%%%%%%%%%%%%%%%%%%%%%%%%%%%%%%%%%%%%%%%%%%%%%%%%%%%%%%%%%%%%%%%%%%%%%
  \item Rappelez la statistique $T_p$ du test de Student pour la nullité du coefficient $\beta_p$,
  et sa loi sous l'hypothèse $\mathcal{H}_{0}$.
  
  \cor{
  D'après le cours:
  $$
  T_p = \frac{\hat{\beta}_p}{\sqrt{\hat{\sigma}^2 [(\mathbf{X}^T\mathbf{X})^{-1}]_{pp}}},
  $$
  avec $\hat{\sigma}^2 = \frac{1}{n - p} \| \mathbf{y} - \mathbf{X}\hat{\boldsymbol{\beta}} \|^2$.
  Sous $\mathcal{H}_0$, $T_p$ suit une loi de Student à $n-p$ degrés de liberté.
  }
  
%%%%%%%%%%%%%%%%%%%%%%%%%%%%%%%%%%%%%%%%%%%%%%%%%%%%%%%%%%%%%%%%%%%%%%%%%%%%%%%%
  \item On décompose $\mathbf{X}$ en blocs:
  $$
  \mathbf{X} = \left[\mathbf{X}_{0} \  \mathbf{X}_{p}\right]
  \quad \text{avec} \quad
  \mathbf{X}_{0} = \left[\mathbf{X}_{1}\ \cdots\ \mathbf{X}_{p-1}\right],
  $$
  où $\mathbf{X}_{0}$ est la matrice de taille $n \times(p-1)$
  des $(p-1)$ premières colonnes de $\mathbf{X}$.
  
  Écrivez les deux modèles emboîtés qui correspondent au test 
  de la nullité du coefficient $\beta_p$.
  Donnez la statistique $F_p$ du test de Fisher correspondant, 
  et sa loi sous $\mathcal{H}_{0}$.
  
  \cor{
  On note
  \begin{align*}
  \text{Modèle 1}&: \mathbf{y} = \mathbf{X}_0 \boldsymbol{\beta}_0 + \boldsymbol{\epsilon}'\\
  \text{Modèle 2}&: \mathbf{y} = \mathbf{X} \boldsymbol{\beta} + \boldsymbol{\epsilon}\\
  \end{align*}
  La statistique de test s'écrit:
  $$
  F_p = \frac{
  \|\hat{\mathbf{y}} -  \mathbf{X}_0\hat{\boldsymbol{\beta}}_0\|^2 / (p - (p - 1))
  }{
  \|\mathbf{y} - \mathbf{X}\hat{\boldsymbol{\beta}}\|^2 / (n - p)
  }
  = \frac{
  \|\hat{\mathbf{y}} -  \mathbf{X}_0\hat{\boldsymbol{\beta}}_0\|^2
  }{
  \|\mathbf{y} - \mathbf{X}\hat{\boldsymbol{\beta}}\|^2 / (n - p)
  }
  $$
  Sous $\mathcal{H}_0$, $F_p$ suit une loi de Fisher à $1$, $n-p$ degrés de libertés. 
  }
  
%%%%%%%%%%%%%%%%%%%%%%%%%%%%%%%%%%%%%%%%%%%%%%%%%%%%%%%%%%%%%%%%%%%%%%%%%%%%%%%%
  \item En utilisant la décomposition $\mathbf{X} = \left[\mathbf{X}_{0} \  \mathbf{X}_{p}\right]$,
  donnez la matrice $\mathbf{X}^T \mathbf{X}$ sous forme de 4 blocs.
  
  \cor{
  $$
  \mathbf{\mathbf{X}^T \mathbf{X}} = 
  \begin{pmatrix}
  \mathbf{\mathbf{X}_0^T \mathbf{X}_0} & \mathbf{\mathbf{X}_0^T \mathbf{X}_p}\\
  \mathbf{\mathbf{X}_p^T \mathbf{X}_0} & \mathbf{\mathbf{X}_p^T \mathbf{X}_p}\\
  \end{pmatrix}
  $$
  }
  
%%%%%%%%%%%%%%%%%%%%%%%%%%%%%%%%%%%%%%%%%%%%%%%%%%%%%%%%%%%%%%%%%%%%%%%%%%%%%%%%
  \item On admet le lemme d'inversion matricielle par blocs suivant:
  
  \textit{
  Soit $\mathbf{M}$ une matrice par blocs,
  $\mathbf{M} = 
  \begin{pmatrix}
  \mathbf{A} & \mathbf{B}\\
  \mathbf{C} & \mathbf{D}\\
  \end{pmatrix},$
  avec $\mathbf{A}$, $\mathbf{B}$, $\mathbf{C}$, $\mathbf{D}$ de dimensions
  respectives $q \times q$, $q \times r$, $r \times q$, et $r \times r$.
  On suppose $\mathbf{M}$ et $\mathbf{A}$ inversibles.
  Alors, on peut écrire $\mathbf{M}^{-1}$ sous la forme:
  $$
  \mathbf{M}^{-1} =
  \begin{pmatrix}
  \mathbf{E} & \mathbf{F}\\
  \mathbf{G} & \mathbf{H}\\
  \end{pmatrix},
  \quad
  \text{avec}
  \quad
  \mathbf{H}^{-1} = \mathbf{D} -  \mathbf{C}\mathbf{A}^{-1}\mathbf{B}.
  $$
  }
  
  Montrez la relation suivante :
  $$
  \frac{1}{[(\mathbf{X}^T\mathbf{X})^{-1}]_{pp}}
  = \mathbf{X}^T_p\mathbf{X}_p
  - \mathbf{X}^T_p\mathbf{X}_0 (\mathbf{X}^T_0\mathbf{X}_0)^{-1} \mathbf{X}^T_0\mathbf{X}_p
  $$
    \cor{
  D'après le lemme d'inversion matricielle:
  $$
  [[(\mathbf{X}^T\mathbf{X})^{-1}]_{pp}]^{-1}
  = \mathbf{X}^T_p\mathbf{X}_p
  - (\mathbf{X}^T_p\mathbf{X}_0) (\mathbf{X}^T_0\mathbf{X}_0)^{-1} (\mathbf{X}^T_0\mathbf{X}_p).
  $$
  }
  
  \item   
  On note $\mathbf{P}_{0}$ la matrice de projection orthogonale sur l'espace
  $\mathcal{M}_{0}$ engendré par les $p - 1$ colonnes de $\mathbf{X}_{0}$,
  et $\mathbf{P}$ la matrice de projection orthogonale sur $\mathcal{M}$
  engendré par les $p$ colonnes de $\mathbf{X}$.
  
  Donnez les expressions de $\mathbf{P}_{0}$ et $\mathbf{P}$ en fonction de, 
  respectivement, $\mathbf{X}_{0}$ et $\mathbf{X}$, puis
  montrez la relation suivante :
  $$
  \frac{1}{[(\mathbf{X}^T\mathbf{X})^{-1}]_{pp}}
  = \mathbf{X}_{p}^T \left(\mathbf{I}_{n} - \mathbf{P}_{0}\right) \mathbf{X}_{p}.
  $$
  \cor{
  On a $\mathbf{P}_{0} = \mathbf{X}_0(\mathbf{X}^T_0\mathbf{X}_0)^{-1} \mathbf{X}^T_0$, d'où:
  $$
  \frac{1}{[(\mathbf{X}^T\mathbf{X})^{-1}]_{pp}}
  = \mathbf{X}^T_p\mathbf{X}_p - \mathbf{X}^T_p \mathbf{P}_{0} \mathbf{X}_p
  = \mathbf{X}^T_p (\mathbf{I}_n - \mathbf{P}_{0}) \mathbf{X}_p.
  $$
  }
  
%%%%%%%%%%%%%%%%%%%%%%%%%%%%%%%%%%%%%%%%%%%%%%%%%%%%%%%%%%%%%%%%%%%%%%%%%%%%%%%%
  \item On décompose $\hat{\boldsymbol{\beta}}$ en deux blocs:
  $
  \hat{\boldsymbol{\beta}} = 
  \begin{pmatrix}
  \hat{\boldsymbol{\beta}}_{0}\\
  \hat{\beta}_p
  \end{pmatrix}.
  $
  Montrez :
  $
  \mathbf{X}\hat{\boldsymbol{\beta}} 
  = \mathbf{X}_0\hat{\boldsymbol{\beta}}_{0} + \mathbf{X}_p\hat{\beta}_p.
  $
  
  \cor{
  $$
  \mathbf{X}\hat{\boldsymbol{\beta}} 
  = (\mathbf{X}_0 \ \mathbf{X}_p)
  \begin{pmatrix}
  \hat{\boldsymbol{\beta}}_{0}\\
  \hat{\beta}_p
  \end{pmatrix}
  = \mathbf{X}_0\hat{\boldsymbol{\beta}}_{0} + \mathbf{X}_p\hat{\beta}_p.
  $$
  }
  
%%%%%%%%%%%%%%%%%%%%%%%%%%%%%%%%%%%%%%%%%%%%%%%%%%%%%%%%%%%%%%%%%%%%%%%%%%%%%%%%
  \item On note $\hat{\mathbf{y}}$ et $\hat{\mathbf{y}}_0$ les projetés orthogonaux
  de $\mathbf{y}$ sur $\mathcal{M}$ et $\mathcal{M}_0$.
  Justifiez l'égalité:
  $$
  \hat{\mathbf{y}}_0 = \mathbf{P}_0\hat{\mathbf{y}}.
  $$
  En déduire:
  $$
  \hat{\mathbf{y}} - \hat{\mathbf{y}}_0 = (\mathbf{I}_n - \mathbf{P}_{0}) \mathbf{X}_p\hat{\beta}_p
  $$
  
  \cor{
  Du fait que les espaces vectoriels sont emboîtés:
  $$
  \hat{\mathbf{y}}_0 = \mathbf{P}_0\mathbf{y} = \mathbf{P}_0\mathbf{P} \mathbf{y} = \mathbf{P}_0\hat{\mathbf{y}}.
  $$
  D'où:
  \begin{align*}
  \hat{\mathbf{y}} - \hat{\mathbf{y}}_0
  &= \hat{\mathbf{y}} - \mathbf{P}_0\hat{\mathbf{y}}
  = (\mathbf{I}_n - \mathbf{P}_{0}) \hat{\mathbf{y}}
  = (\mathbf{I}_n - \mathbf{P}_{0}) \mathbf{X}\hat{\boldsymbol{\beta}}\\
  &= (\mathbf{I}_n - \mathbf{P}_{0}) (\mathbf{X}_0\hat{\boldsymbol{\beta}}_{0} + \mathbf{X}_p\hat{\beta}_p)
  = (\mathbf{I}_n - \mathbf{P}_{0}) \mathbf{X}_p\hat{\beta}_p.
  \end{align*}
  }
  
%%%%%%%%%%%%%%%%%%%%%%%%%%%%%%%%%%%%%%%%%%%%%%%%%%%%%%%%%%%%%%%%%%%%%%%%%%%%%%%%
  \item Montrez que $T_p^2 = F_p$. En déduire l'équivalence des deux tests.
  
  \cor{
  \begin{align*}
  T_p^2
  & = \frac{\hat{\beta}_p^2}{\hat{\sigma}^2 [(\mathbf{X}^T\mathbf{X})^{-1}]_{pp}} 
   = \frac{\mathbf{X}^T_p (\mathbf{I}_n - \mathbf{P}_{0}) \mathbf{X}_p \hat{\beta}_p^2}{\hat{\sigma}^2}\\
  & = \frac{\|(\mathbf{I}_n - \mathbf{P}_{0}) \mathbf{X}_p \hat{\beta}_p\|^2}{\hat{\sigma}^2} 
   = \frac{\|\hat{\mathbf{y}} - \hat{\mathbf{y}}_0\|^2}{\hat{\sigma}^2} \\
  &= F_p
  \end{align*}
  
  Les deux tests sont donc bien équivalents.
  }
\end{enumerate}

% 6. En notant $\widehat{Y}$ et $\widehat{Y}_{0}$ les projetés orthogonaux de $Y$ sur $\mathcal{M}$ et $\mathcal{M}_{0}$, et après avoir justifié le fait que
% $$
% \hat{Y}_{0}=P_{0} Y=P_{0} P Y=P_{0} \hat{Y}
% $$
% montrer que
% $$
% \widehat{Y}-\widehat{Y}_{0}=\widehat{\beta}_{p}\left(I_{n}-P_{0}\right) \mathbf{X}_{p}
% $$
% 7. En déduire que $F=T^{2}$.

%%%%%%%%%%%%%%%%%%%%%%%%%%%%%%%%%%%%%%%%%%%%%%%%%%%%%%%%%%%%%%%%%%%%%%%%%%%%%%%%%
		\exo{Estimation sous contrainte}
%%%%%%%%%%%%%%%%%%%%%%%%%%%%%%%%%%%%%%%%%%%%%%%%%%%%%%%%%%%%%%%%%%%%%%%%%%%%%%%%%
		Dans le modèle de régression linéaire,
		il arrive parfois que l'on souhaite imposer des contraintes linéaires à $\beta$,
		par exemple que sa première coordonnée soit égale à $1$.
		Nous supposerons en général que nous imposons $q$ contraintes linéairement indépendantes à $\beta$,
		ce qui s'écrit sous la forme : $R \beta=r$,
		où $R$ est une matrice $q \times p$ de rang $q<p$ et $r$ un vecteur de taille $q$.
		Montrer que l'estimateur des moindres carrés sous contraintes s'écrit:
	\[
	\hat \beta_c = \hat \beta + (X^T X)^{-1} R^T  \big[ R (X^T X)^{-1} R^T \big]^{-1} (r - R \hat \beta)\,.
	\]
	Montrez qu'il est sans biais, et que sa variance est égale à :
	\[
		\mathbf{V}[\hat\beta_c] = \mathbf{V}[\hat\beta] - \sigma^2 (X^T X)^{-1} R^T (R (X^T X)^{-1} R^T)^{-1} R (X^T X)^{-1}.
	\]
	\cor{
	On résaud le problème d'optimisation sous contrainte :
	\[
	\hat \beta_c = \text{argmin}_{\beta \in \R^p} \|Y - X \beta\|^2 \quad \text{sous contrainte de} \quad R \beta - r = 0_q
	\]
	On utilise la techinique du lagrangien, en introduisant le paramètre $\lambda \in \R^q$ :
	\[
	L(\beta, \lambda) 
	= \|Y - X \beta\|^2 - \lambda^T (R \beta - r)
	= Y^T Y - 2 \beta^T X^T Y  + \beta^T X^T X \beta - \lambda^T R \beta + \lambda^T r
	\]
	On cherche ensuite les points critiques du lagrangien :
	\[
	\begin{aligned}
	\left\{
	\begin{aligned}
	\nabla_{\beta} L(\beta, \lambda) &= 0 \\
	\nabla_{\lambda} L(\beta, \lambda) &= 0
	\end{aligned}
	\right.
	& \iff
	\left\{
	\begin{aligned}
	- 2 X^T Y + 2 X^T X \hat\beta_c - R^T \hat\lambda &= 0 & (*)\\
	- R \hat\beta_c + r &= 0
	\end{aligned}
	\right.
	\\
	& \iff
	\left\{
	\begin{aligned}
	R (X^T X)^{-1} (- 2 X^T Y + 2 X^T X \hat\beta_c - R^T \hat\lambda) &= 0 \\
	R \hat\beta_c &= r
	\end{aligned}
	\right.
	\\
		& \iff
	\left\{
	\begin{aligned}
	R (X^T X)^{-1} R^T \hat\lambda &= -2 R (X^T X)^{-1} X^T Y + 2 R \hat\beta_c \\
	R \hat\beta_c &= r
	\end{aligned}
	\right.
	\\
			& \iff
	\left\{
	\begin{aligned}
	\hat\lambda &= 2 (R (X^T X)^{-1} R^T)^{-1} (r - R (X^T X)^{-1} X^T Y) \\
	R \hat\beta_c &= r
	\end{aligned}
	\right.
	\\
	\end{aligned}
	\]
	En ré-injectant $\hat\lambda$ dans $(*)$, on trouve :
	\begin{align*}
	 \hat\beta_c 
	 & = (X^T X)^{-1} X^T Y + \frac{1}{2} (X^T X)^{-1} R^T \hat\lambda \\
	 & = (X^T X)^{-1} X^T Y + (X^T X)^{-1} R^T (R (X^T X)^{-1} R^T)^{-1} (r - R (X^T X)^{-1} X^T Y) \\
	 & = \hat\beta + (X^T X)^{-1} R^T (R (X^T X)^{-1} R^T)^{-1} (r - R \hat\beta)
	\end{align*}
	Puis :
	\begin{align*}
	\mathbf{E}[\hat\beta_c]
	&= \mathbf{E}[\hat\beta + (X^T X)^{-1} R^T (R (X^T X)^{-1} R^T)^{-1} (r - R \hat\beta)] \\
	&= \beta + (X^T X)^{-1} R^T (R (X^T X)^{-1} R^T)^{-1} (r - R \beta) \\
	&= \beta
	\end{align*}
	Et:
	\begin{align*}
	\mathbf{V}[\hat\beta_c]
	&= \mathbf{V}[\hat\beta + (X^T X)^{-1} R^T (R (X^T X)^{-1} R^T)^{-1} (r - R \hat\beta)] \\
	&= \mathbf{V}[\hat\beta - (X^T X)^{-1} R^T (R (X^T X)^{-1} R^T)^{-1} R \hat\beta] \\
	&= \mathbf{V}[(I_p - B) \hat\beta] & B = (X^T X)^{-1} R^T (R (X^T X)^{-1} R^T)^{-1} R \\
	&= (I_p - B) \mathbf{V}[\hat\beta] (I_p - B)^T \\
	&= (I_p - B) \sigma^2 (X^T X)^{-1} (I_p - B)^T \\
	&= \sigma^2 \left[(X^T X)^{-1} - B (X^T X)^{-1} -  (X^T X)^{-1} B^T + B (X^T X)^{-1} B^T  \right]\\
	\end{align*}
	Mais :
	\begin{align*}
	B (X^T X)^{-1} B^T
	& = (X^T X)^{-1} R^T (R (X^T X)^{-1} R^T)^{-1} R (X^T X)^{-1} R^T (R (X^T X)^{-1} R^T)^{-1} R (X^T X)^{-1} \\
	& = (X^T X)^{-1} R^T (R (X^T X)^{-1} R^T)^{-1} R (X^T X)^{-1} \\
	& = B (X^T X)^{-1}
	\end{align*}
	Donc :
	\begin{align*}
	\mathbf{V}[\hat\beta_c]
	&= \sigma^2 \left[(X^T X)^{-1} - (X^T X)^{-1} B^T \right]\\
	&= \mathbf{V}[\hat\beta] - \sigma^2 (X^T X)^{-1} R^T (R (X^T X)^{-1} R^T)^{-1} R (X^T X)^{-1}.
	\end{align*}
	}
	
	
\end{document}


